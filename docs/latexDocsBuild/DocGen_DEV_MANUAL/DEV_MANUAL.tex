% ===================================================== %
% ===================  PREAMBULE ====================== %
% ===================================================== %

\documentclass[a4paper, french]{report}
\usepackage{config}

% ===================================================== %
% ===================================================== %
% ===================================================== %

\begin{document}

% Page de titre standarde
\begin{titlepage}
    \begin{flushleft}
        \includegraphics[width=5cm]{UP.png}\par
        \centering
        
        \vspace{13\baselineskip}       
        \HRule \\[0.4cm]

        {\Huge 
        Manuel du développeur de DocGen\par}
        \vspace{0.4cm}
        \HRule
        \vfill
      
        Auteur(s): Florian Legendre\medskip \par
        
        \includegraphics[scale=0.7]{creative_commons.png}\par
    \end{flushleft}
\end{titlepage}

% Explication des légendes et abbréviations Standard
\newpage
\begin{LARGE}
Légendes et Abbréviations utilisées\\\\\\\\
\end{LARGE}

\begin{lstlisting}[style=C, caption=Exemple de code source]
Ceci est du code source.
Selon les langages, différents mots seront colorés selon 
si ce sont des mots clefs ou non (comme int, char, etc.).
\end{lstlisting}

\begin{mdframed}[style=Bash]
\begin{lstlisting}[style=Bash, caption=Exemple d'une pseudo capture d'écran Bash]
Ceci est un formattage automatique Latex d'un texte copié-collé
directement depuis un terminal Bash ayant valeur de capture
d'écran. La coloration correspond à une coloration quelconque 
d'un terminal Bash (les chemins étant habituellement coloré et 
le nom de l'utilisateur aussi comme crex@crex:~$ ...)
\end{lstlisting}
\end{mdframed}


% Sommaire Standard
\newpage
\pagestyle{empty}
\setcounter{tocdepth}{3}
\tableofcontents
\addtocontents{toc}{\protect\thispagestyle{empty} 
                    \protect\pagestyle{empty}}
\pagestyle{plain}

\newpage



%%% ---------------------------------------------- %%%
%%% ------------------ Début TP ------------------ %%%
%%% ---------------------------------------------- %%%

\part{Cahier des Charges}
\subfile{./contents/1_cahierDesCharges.tex}

\part{Spécifications Fonctionnelles}
\subfile{./contents/2_specFonc.tex}

\part{Bibliographie et Glossaires}
\subfile{./contents/3_biblio.tex}

\part{Annexes}
\subfile{./contents/4_annexes.tex}

\end{document}