\documentclass[./standalone.tex]{subfiles}
%\documentclass[../../../CR/pac.tex]{subfiles}

\begin{document}

Pour installer DocGen:
\begin{enumerate}
    \item Téléchargez le projet sur sa page SourceSup (cf. figure     \ref{fig:installation1} ci-dessous) en utilisant le logiciel \textit{git bash} et en y entrant la commande indiquée dans le dossier de votre choix
    \item Supprimez le dossier \textit{.git/} associé au projet (le fichier est peut-être caché, cf. figure \ref{fig:installation3} ci-dessous)
    \item Copiez-collez le dossier docgen/ dans votre projet (idéalement dans un dossier lib/lib\_ext/ mais vous avez probablement votre arborescence à vous) et passez à la configuration!\\
\end{enumerate}
\vspace{1cm}

\begin{figure}[h!]
    \centering
    \includegraphics[scale=0.3]{images/installation1.png}
    \caption{Téléchargement du projet soit par le lien 1 (si vous êtes collaborateur vous ne pourrez pas collaborer avec ce lien), soit par le lien 2}
    \label{fig:installation1}
\end{figure}
\vspace{1cm}

\begin{figure}[h!]
    \centering
    \includegraphics[scale=0.45]{images/installation2.png}
    \caption{Résultat du téléchargement et point d'entrée de l'utilisation de DocGen}
    \label{fig:installation2}
\end{figure}
\vspace{1cm}

\begin{figure}[h!]
    \centering
    \includegraphics[scale=0.45]{images/installation3.png}
    \caption{Le dossier \textit{.git/} à supprimer}
    \label{fig:installation3}
\end{figure}


\end{document}